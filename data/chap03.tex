\chapter{陆地生态系统碳循环模型开放式对比框架}
% MDL 模型描述语言
% UDX 统一数据交换表达模型
% UDX JSON Schema
基于Web Service的开放式模型对比框架和传统模型对比框架最大的区别就是,前者通过Web Service提供开放式的可共享与重用的模型服务资源和数据服务资源,因此模型和数据资源的开放式接入方法是在网络空间下开展地理模型对比的前提。本章首先分析了地理模型资源的运行特征和描述方法,并在此基础上阐述了模型资源的开放式接入方法;其次讨论了模型所依赖的数据资源的特征及元数据描述方法,并在其上构建了数据资源的开放式接入方法;最后探讨了如何对异构的地理模型资源和数据资源进行数据适配和耦合。

\section{开放式对比框架需求分析}
% 对比情景总结 参考CMIP5/CMIP6


\section{开放式对比情景分析和总结}
% topic solution task
% 可共享 可重用

\section{基于组件的开放式模型对比框架设计}

\subsection{模型资源库}
\subsection{数据资源库}
\subsection{标准度量库}
\subsection{数据重构方法库}
\subsection{可视化方法库}
\subsection{对比方法库}

% \section{面向对比的标准度量研究}
% \subsection{面向对比的标准度量及其描述方法}
% \subsection{面向标准度量的数据适配方法}

% \section{模型对比科学工作流引擎}
% % 模型调用-数据缓存-数据重构-数据对比
% \subsection{对比流程分析和归纳}
% \subsection{对比自动化执行引擎}

% \subsection{面向对比的开放式地理资源库}
% \subsubsection{模型资源库}
% % 介绍模型
% \begin{enumerate}[(1)]
% \item \textbf{IBIS}

% \item \textbf{Biome-BGC}

% \item \textbf{LPJ}

% \end{enumerate}


% \subsubsection{数据资源库}
% % 介绍数据
% \begin{enumerate}[(1)]
% \item \textbf{气象数据集}

% \item \textbf{土壤数据集}

% \item \textbf{Fluxdata观测数据集}

% \item \textbf{MODIS GPP产品数据集}

% \item \textbf{模型输出数据集}

% \end{enumerate}
% \subsubsection{标准度量库}
% \subsubsection{数据重构方法库}
% \subsubsection{可视化方法库}
% \subsubsection{对比方法库}
% 介绍对比方法
\begin{enumerate}[(1)]
\item \textbf{泰勒图}

\item \textbf{时间序列折线图}

\item \textbf{箱图}

\item \textbf{热力图}

\item \textbf{偏差等值线图}

\item \textbf{加权超级集合}

\end{enumerate}

\section{本章小结}


