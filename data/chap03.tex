\chapter{模型和数据资源开放式接入方法}
% MDL 模型描述语言
% UDX 统一数据交换表达模型
% UDX JSON Schema
基于Web Service的开放式模型对比框架和传统模型对比框架最大的区别就是,前者通过Web Service提供开放式的可共享与重用的模型服务资源和数据资源,因此模型和数据资源的开放式接入方法是在网络空间下开展模型对比的前提。本章首先分析了地理模型资源的运行特征,再次基础上阐述了模型资源的开放式接入方法;其次讨论了模型所依赖的数据资源的元数据描述方法,并在其上构建了数据资源的开放式接入方法;最后探讨了如何对异构的地理模型资源和数据资源进行数据适配和耦合。

\section{模型资源的开放式接入方法}


\subsection{地理模型运行特征分析}
\subsection{地理模型描述接口设计}
\subsection{模型资源的封装和服务发布}

\section{数据资源的开放式接入方法}
\subsection{基于UDX Schema的数据规格描述方法}
% \subsection{标准数据集}
% \subsection{数据处理和可视化方法}
\subsection{数据资源的服务发布}

\section{模型资源和数据资源的耦合方法}
\subsection{基于JSON Schema的数据描述与验证}
\subsection{数据重构}

\section{本章小结}
