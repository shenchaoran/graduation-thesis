\chapter{开放式模型和数据资源接入方法}
% MDL 模型描述语言
% UDX 统一数据交换表达模型
% UDX JSON Schema
基于Web Service的开放式模型对比框架和传统模型对比框架最大的区别就是,前者通过Web Service提供开放式的可共享与重用的模型服务资源和数据服务资源,因此模型和数据资源的开放式接入方法是在网络空间下开展地理模型对比的前提。本章首先分析了地理模型资源的运行特征和描述方法,并在此基础上阐述了模型资源的开放式接入方法;其次讨论了模型所依赖的数据资源的特征及元数据描述方法,并在其上构建了数据资源的开放式接入方法;最后探讨了如何对异构的地理模型资源和数据资源进行数据适配和耦合。

\section{开放式模型资源接入方法}
% 从支撑模型运行的角度出发,并结合地理模型对比的需求,梳理了地理模型的运行特征,并以结构化的文档描述模型资源,

\subsection{地理模型资源运行特征分析}
% 运行分类:简单型、时间推进型、循环迭代型

\subsection{地理模型资源描述方法}
\subsection{模型资源的封装和服务发布}

\section{开放式数据资源接入方法}
\subsection{地理数据资源结构特征分析}
\subsection{地理数据资源结构化描述方法}
% \subsection{标准数据集}
% \subsection{数据处理和可视化方法}
\subsection{地理数据资源的服务化封装}

\section{模型资源和数据资源的耦合方法}
\subsection{基于UDX Schema的数据规格检查与匹配}
\subsection{基于数据抽取和重构的耦合方法}

\section{本章小结}
