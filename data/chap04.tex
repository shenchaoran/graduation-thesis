\chapter{开放式地理模型对比方案研究}
% 对比描述语言: CDL
% 对比度量标准: Metric
% 简介模型对比流程:话题-方案-任务

\section{开放式地理模型资源接入方法}
% 从支撑模型运行的角度出发,并结合地理模型对比的需求,梳理了地理模型的运行特征,并以结构化的文档描述模型资源,

\subsection{地理模型资源运行特征分析}
% 运行分类:简单型、时间推进型、循环迭代型

\subsection{地理模型资源描述方法}
\subsection{模型资源的封装和服务发布}

\section{开放式地理数据资源接入方法}
\subsection{地理数据资源结构特征分析}
\subsection{地理数据资源结构化描述方法}
% \subsection{标准数据集}
% \subsection{数据处理和可视化方法}
\subsection{地理数据资源的服务化封装}

% \section{模型资源和数据资源的耦合方法}
% \subsection{基于UDX Schema的数据规格检查与匹配}
% \subsection{基于数据抽取和重构的耦合方法}

\section{开放式地理模型对比方法}
% 模型对比方法,第二章已经介绍过了。这里列一下就行了
% \subsection{模型对比方法}
% \subsubsection{泰勒图}
% \subsubsection{}
\subsection{地理模型对比方法总结与归纳}
% \subsection{统计学对比方法}
% \subsection{可视化对比方法}
\subsection{开放式模型对比方法接入方法}

\section{本章小结}
