\chapter{开放式对比资源接入方法}
\section{开放式地理数据资源接入方法}
\subsection{地理数据资源结构特征分析}
\subsection{地理数据资源结构化描述方法}
% \subsection{标准数据集}
\subsection{地理数据资源的服务化封装}

\section{开放式地理模型资源接入方法}
% 从支撑模型运行的角度出发,并结合地理模型对比的需求,梳理了地理模型的运行特征,并以结构化的文档描述模型资源,

\subsection{地理模型资源运行特征分析}
% 运行分类:简单型、时间推进型、循环迭代型
生态过程模型普遍采用植被功能类型(PFT)作为植被基本处理单元,通常将PFT划分为林地(覆盖热带、温带、寒带,常绿、落叶、针叶和阔叶等)、灌木(常绿和落叶等)、草地($C_3$、$C_4$等)、农田(水稻、玉米、温带谷物、大豆、热带根系作物、太阳花、花生、油菜等)等类型。每种植被功能类型对应有许多植被生理生态参数。

\subsection{地理模型资源描述方法}
\subsection{模型资源的封装和服务发布}

% \section{模型资源和数据资源的耦合方法}
% \subsection{基于UDX Schema的数据规格检查与匹配}
% \subsection{基于数据抽取和重构的耦合方法}

\section{开放式地理模型对比方法}
\subsection{模型对比方法}
\begin{enumerate}[(1)]
\item \textbf{泰勒图}

\item \textbf{时间序列折线图}

\item \textbf{箱图}

\item \textbf{热力图}

\item \textbf{偏差等值线图}

\item \textbf{加权超级集合}

\end{enumerate}

\subsection{地理模型对比方法总结与归纳}
\begin{enumerate}[(1)]
\item \textbf{统计学对比方法}

\item \textbf{可视化对比方法}
\end{enumerate}

\subsection{开放式模型对比方法接入方法}

\section{本章小结}
