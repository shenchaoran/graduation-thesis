\chapter{陆地生态系统碳循环模型对比场景分析}
% - 章节意义:...是...的重点/前提,...有重大意义。
% - 本章内容:本章首先分析/阐述...,进一步...;接着探讨...;最后讨论...。针对框架构建的关键问题,分析现有框架设计的方法及其存在问题,提出了本文的重点内容。

% 不知道哪里改写具体的碳循环模型和数据,哪里该写广义上的地理模型和地理数据

模型对比场景是开展模型对比工作的承载容器。在地理模型对比场景中,模型资源和数据资源作为最重要的两个要素,为开展详细的对比工作准备。
% 模型对比框架的设计是填充模型资源和数据资源,开展详细对比工作的前提。分析现有的对比框架可以为开放式的模型对比框架提供参考和借鉴,找出模型对比框架设计的重点。

本章首先分析了地理模型资源和数据资源的多源异构复杂性和这种复杂性对开展模型对比工作的阻碍;然后分析和归纳了现阶段国内外主流的模型对比方案设计方法;最后探讨了在开放式网络环境下如何设计公开化可参与的对比框架,并讨论了模型对比过程中几个至关重要的地理资源库。

\section{陆地生态系统碳循环模型及其特征分析}
\subsection{陆地生态系统碳循环模型及其分类}
陆地、大气、海洋以及化石燃料碳库是地球系统碳循环的四个主要组成部分,其中陆地生态系统又是其中最为活跃的碳库,也是人类活动聚集的场所。陆地生态系统碳循环模型通过模拟植物光合器官碳库、植物支持器官碳库、凋落物碳库和土壤有机碳库之间的碳源汇交换,来计算植被的初级生产力(毛留喜,2006)。

陆地生态系统碳循环模型从机理上可分为统计模型、生态过程模型和遥感、过程耦合模型,如图~\ref{fig:carbon-model-class}所示。统计模型可分为气候统计模型和遥感统计模型。其中气候统计模型主要通过在气候因子与植被净初级生产力(Net Primary Productivity,NPP)的实测数据之间建立回归方程;遥感统计模型通过遥感光谱指数(如NDVI)与NPP、生物量等数据间的相关关系进行统计回归。统计模型简单直观,具有较强的区域适用性,但其完全依赖于地面观测数据,对于不同的区域,模型不具备普适性和推广性。同时统计模型没有考虑陆地生态系统碳循环过程的内部机理,无法揭示生态系统与环境间的相互影响关系,不能用于对未来的预测研究(袁文平,2014;谢馨瑶,2018)。生态过程模型按照是否考虑实际环境对植被类型、组成和结构的影响分为基于动态植被类型的模型和基于静态植被类型的模型(王绍刚,2008;王萍,2009;毛留喜,2006),按照涉及到的机理类型分为地球化学过程模型、陆面物理过程模型和生物过程模型(谢馨瑶,2018)。过程模型由于其综合考虑了碳循环过程的动力学特征,结合了气候、土壤和植被生理生态参数,以及陆地生态系统与大气、海洋之间的相互作用,模拟结果相对来说更加准确,逐渐占据了主导地位。遥感、过程耦合模型通过将遥感观测数据(如叶面积指数LAI)同化到模型之中,来提高模型模拟的精度(张延龙,2015),他融合了遥感统计模型和生态过程模型的优点,可以反映区域和全球尺度的NPP空间分布和变化(朱文泉,2005)。

\begin{figure}
    \centering
    \includegraphics[width=0.9\textwidth]{carbon-model-class}
    \caption{碳循环模型分类}
    \label{fig:carbon-model-class}
\end{figure}

\subsection{陆地生态系统碳循环模型的特点}
% 依赖的数据特点
% 过程拆解

\subsection{IBIS, Biome-BGC, LPJ模型简介}
% 主要介绍这三个模型为何作为该类模型的代表

\section{陆地生态系统碳循环数据资源及其特征分析}
\subsection{陆地生态系统碳循环模型相关数据}
\subsection{数据特征分析}
% \subsection{气象数据集}
% \subsection{土壤数据集}
% \subsection{通量观测数据集}
% \subsection{其他数据集}
% \subsection{数据特征分析}

\section{陆地生态系统碳循环模型对比方案分析和归纳}
\subsection{基于实验协议的模型对比方案}
% CMIP

\subsection{可迁移、复用与重现的模型对比方案}
% CORDEX
% 基于虚拟机镜像/包管理器/docker

\subsection{陆地生态系统碳循环模型对比方案总结}
% 这一部分很重要,要明确的说明以上方案的不足之处
% 引出我们的对比方案的优点和设计
% 这是立题之本

传统的模型对比框架是以数据为基础,不考虑生产数据即模型模拟的过程。

\section{本章小结}