\chapter{开放式地理模型对比场景分析}
% - 章节意义:...是...的重点/前提,...有重大意义。
% - 本章内容:本章首先分析/阐述...,进一步...;接着探讨...;最后讨论...。针对框架构建的关键问题,分析现有框架设计的方法及其存在问题,提出了本文的重点内容。

% 不知道哪里改写具体的碳循环模型和数据,哪里该写广义上的地理模型和地理数据

模型对比场景是开展模型对比工作的承载容器。在地理模型对比场景中,模型资源和数据资源作为最重要的两个要素,为开展详细的对比工作准备。
% 模型对比框架的设计是填充模型资源和数据资源,开展详细对比工作的前提。分析现有的对比框架可以为开放式的模型对比框架提供参考和借鉴,找出模型对比框架设计的重点。

本章首先分析了地理模型资源和数据资源的多源异构复杂性和这种复杂性对开展模型对比工作的阻碍;然后分析和归纳了现阶段国内外主流的模型对比方案设计方法;最后探讨了在开放式网络环境下如何设计公开化可参与的对比框架,并讨论了模型对比过程中几个至关重要的地理资源库。

\section{地理模型和数据资源分析}

\subsection{地理模型资源分析}
% 章节标题用不用改成碳循环模型分析

\subsection{地理数据资源分析}


\section{地理模型对比方案分析和归纳}
\subsection{基于实验协议的模型对比方案}
% CMIP

\subsection{可迁移、复用与重现的模型对比方案}
% CORDEX
% 基于虚拟机镜像/包管理器/docker

\subsection{地理模型对比方案总结}
传统的模型对比框架是以数据为基础,不考虑生产数据即模型模拟的过程。

\section{开放式的模型对比框架设计}
\subsection{总体架构}
\subsection{面向对比的开放式地理资源库}
\subsubsection{模型资源库}
% 介绍模型
\begin{enumerate}[(1)]
\item \textbf{IBIS}

\item \textbf{Biome-BGC}

\item \textbf{LPJ}

\end{enumerate}


\subsubsection{数据资源库}
% 介绍数据
\begin{enumerate}[(1)]
\item \textbf{气象数据集}

\item \textbf{土壤数据集}

\item \textbf{Fluxdata观测数据集}

\item \textbf{MODIS GPP产品数据集}

\item \textbf{模型输出数据集}

\end{enumerate}
\subsubsection{标准度量库}
\subsubsection{数据重构方法库}
\subsubsection{可视化方法库}
\subsubsection{对比方法库}
% 介绍对比方法
\begin{enumerate}[(1)]
\item \textbf{泰勒图}

\item \textbf{时间序列折线图}

\item \textbf{箱图}

\item \textbf{热力图}

\item \textbf{偏差等值线图}

\item \textbf{加权超级集合}

\end{enumerate}

\section{本章小结}