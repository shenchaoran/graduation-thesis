\chapter{陆地生态系统碳循环模型对比分析}
\label{chap:model}
% - 章节意义:...是...的重点/前提,...有重大意义。
% - 本章内容:本章首先分析/阐述...,进一步...;接着探讨...;最后讨论...。针对框架构建的关键问题,分析现有框架设计的方法及其存在问题,提出了本文的重点内容。

\section{陆地生态系统碳循环模型}
% 总的部分介绍简洁一点
% 介绍基本结构、特点、分类
\label{sec:model}
\subsection{陆地生态系统碳循环}
% 介绍主要过程:光合、呼吸等
% 及影响因素
地球系统碳循环是指碳在岩石圈、水圈、大气圈和生物圈之间,以各种形式相互转换和迁移的过程。如图~\ref{fig:carbon-cycling-structure}所示,碳元素主要分布于陆地、海洋、大气和化石燃料四个碳库,其中陆地生态系统中的碳库是含量最小的碳库,但由于其是人类活动聚集的场所,所以又是最为活跃的碳库。它主要由植物光和器官碳库、植物支持器官碳库、凋落物碳库和土壤有机物碳库四部分组成,如图~\ref{fig:land-carbon-cycling-structure}所示,植物通过光合作用将大气中的$CO_2$转换为碳水化合物,再通过自养呼吸、分配光合作用产物和死亡凋落过程把碳输送到大气、植物支持器官碳库和凋落物碳库;植物支持器官通过自养呼吸和死亡凋落把有机碳分别输送到大气和凋落物碳库;凋落物碳库通过异样呼吸和分解作用分别把碳输送到大气和土壤有机物碳库中;土壤有机物碳库最后通过异样呼吸把碳输送回大气碳库中~\cite{2004-esc}~\cite{2006-maoliuxi}。碳在陆地生态系统中的转移路径为:大气——植物光合器官——植物支持器官——凋落物——土壤——大气,形成了一个首尾相连的闭环,因此,可以把陆地生态系统中碳的转移看成一个循环过程。

\begin{figure}[!htbp]
    \centering
    \subcaptionbox{地球系统碳循环基本框架\label{fig:carbon-cycling-structure}}{\includegraphics[width=.2\textwidth]{carbon-cycling-structure}}
    \hfill
    \subcaptionbox{陆地生态系统碳循环基本框架\label{fig:land-carbon-cycling-structure}}{\includegraphics[width=.7\textwidth]{land-carbon-cycling-structure}} \\
    \caption{碳循环框架}
    \label{fig:carbon-cycling}
\end{figure}

\subsection{陆地生态系统碳循环模型及其分类}
陆地生态系统碳循环中碳主要在植被——大气——土壤碳库之间转移,模型一般包括植物的光合作用、自养呼吸和异氧呼吸以及土壤的呼吸作用这四个过程,影响这些过程的因素有大气$CO$浓度、辐射、气温、土壤湿度、叶片中的叶绿素含量,所以许多模型耦合了水循环过程和氮循环过程。陆地生态系统碳循环模型通过生物生产力来描述碳通量的变化,这些指标主要有总初级生产力(Gross Ecosystem Production,GPP)、净第一性生产力(Net Primary production,NPP)、净生态系统生产力(Net Ecosystem Production,NEP),他们之间的关系如公式~\ref{eq:NPP}和~\ref{eq:NEP}所示,式中,$R_A$表示自养呼吸,$R_H$表示异氧呼吸。

\begin{align}
    & NPP = GPP-R_A
    \label{eq:NPP} \\
    & NEP = \left(GPP-R_A\right)-R_H = NPP-R_H
    \label{eq:NEP}
\end{align}

除了这些指标外,模型计算过程中一般还包括有生物量(Biomass)、蒸散发(Evapotranspiration,ET)、径流(Runoff)、叶面积指数(LAI)等指标。

\begin{figure}[!htbp]
    \centering
    \includegraphics[width=0.9\textwidth]{carbon-model-class}
    \caption{碳循环模型分类}
    \label{fig:carbon-model-class}
\end{figure}

陆地生态系统碳循环模型从机理上可分为统计模型、生态过程模型和遥感、过程耦合模型,如图~\ref{fig:carbon-model-class}所示。
统计模型可分为气候统计模型和遥感统计模型。其中气候统计模型主要通过在温度、降水等气候因子与植被净初级生产力(Net Primary Productivity,NPP)的实测数据之间建立回归方程,代表性的有Miami模型(曼阿密模型)~\cite{lieth1975modeling}、Chikugo模型(筑后模型)~\cite{seino1985agroclimatic}和Thornth-waite Memorial模型(桑斯维特纪念模型)~\cite{hai2012change};
遥感统计模型是基于遥感光谱指数建立的模型。由于总初级生产力(GPP)与截留光合有效辐射量和辐射的光合利用效率(LUE)成正比,而截留光合有效辐射与总光合有效辐射(PAR)和光合有效辐射截留率(FPAR)成正比。其中PAR可以通过气候学方法计算得到,FPAR可以通过遥感方法得到,因此,GPP可以表达为公式~\ref{eq:GPP},NPP可以表达为公式~\ref{eq:NPP}

\begin{align}
    &GPP(FPAR, PAR) = FPAR \cdot PAR \cdot LUE \cdot f(T, W, CO_2)
    \label{eq:GPP}  \\
    &NPP = GPP - R_A
    \label{eq:NPP}
\end{align}

式中,$f(T, W, CO_2)$表示温度T、土壤湿度W和大气中$CO_2$浓度对辐射的诗集光合利用效率的影响;$R_A$表示自养呼吸;代表性的模型有CASA、GLO-PEM、TURC、SIB2等
% 遥感统计模型主要通过遥感光谱指数(如NDVI)与植被净初级生产力、生物量等数据间的相关关系进行统计回归。
统计模型简单直观,具有较强的区域适用性,但其完全依赖于地面观测数据,对于不同的区域,模型不具备普适性和推广性。同时统计模型没有考虑陆地生态系统碳循环过程的内部机理,无法揭示生态系统与环境间的相互影响关系,不能用于对未来的预测研究~\cite{2014-yuanwenping}~\cite{xieqinyao}。

生态过程模型按照是否考虑实际环境对植被功能类型、组成和结构的影响分为基于动态植被类型的模型和基于静态植被类型的模型~\cite{2008-wangshaogang}~\cite{2009-wangping}~\cite{2006-maoliuxi},按照涉及到的机理类型分为地球化学过程模型、陆面物理过程模型和生物过程模型~\cite{xieqinyao}。过程模型由于其综合考虑了碳循环过程的动力学特征,结合了气候、土壤和植被生理生态参数,以及陆地生态系统与大气、海洋之间的相互作用,模拟结果相对来说更加准确,逐渐占据了主导地位。图~\ref{fig:DGVMs-structure}是生态过程模型的典型结构,其中冠层物理主要包括辐射传输、能量平衡、水平衡和气溶胶动力等过程,冠层生理主要用于计算光合作用和气孔导度,土壤物理包括能量平衡、水平衡和土壤温度等过程,自养呼吸主要指植被的维持呼吸和生长呼吸,异氧呼吸则指土壤呼吸和枯枝层分解,植被生理主要包括同化物分配和组织流转等过程,营养循环主要指碳、氮循环,植被竞争包括光竞争、水竞争和空间竞争,干扰过程主要涉及火灾、采伐( 或收割) 、林隙、热胁迫、干旱胁迫、背景胁迫( 如接近植被生命年限,死亡不可避免) 等,萌生过程主要指植被幼苗或种子在自然环境下的成活或萌发过程。生态过程模型的输入数据一般包括气象数据、$CO_2$浓度数据、土壤质地数据、高程数据和植被功能类型数据等。

生态过程模型普遍采用植被功能类型(PFT)作为植被基本处理单元,通常将PFT划分为林地(覆盖热带、温带、寒带,常绿、落叶、针叶和阔叶等)、灌木(常绿和落叶等)、草地($C_3$、$C_4$等)、农田(水稻、玉米、温带谷物、大豆、热带根系作物、太阳花、花生、油菜等)等类型。每种植被功能类型对应有许多植被生理生态参数,如气孔导度、光合途径、叶片厚度等。

\begin{figure}[!htbp]
    \centering
    \includegraphics[width=0.9\textwidth]{DGVMs-structure}
    \caption{生态过程模型的典型结构}
    \label{fig:DGVMs-structure}
\end{figure}

遥感、过程耦合模型通过将遥感观测数据(如叶面积指数LAI)同化到模型之中,来提高模型模拟的精度(张延龙,2015),他融合了遥感统计模型和生态过程模型的优点,可以反映区域和全球尺度的NPP空间分布和变化~\cite{zhuwenquan-2005}。但由于遥感、过程耦合模型难度过大,在应用时不太方便。

综合分析三类模型的优缺点后,本文选取了IBIS、Biome-BGC、LPJ三个具有代表性的生态过程模型,在全球尺度上进行模拟和对比分析。

\subsection{IBIS, Biome-BGC, LPJ模型简介}
\subsubsection{IBIS模型}
% 起源、原理、结构、适用条件、输入数据、运行模式
IBIS全称为集成生物圈模拟器(Integrated BIosphere Simulator),由美国威斯康星——麦迪逊大学Foley等~\cite{foley1996integrated}提出,随后,Kucharik等~\cite{Kucharik2000Testing}对其进行了改进,现在最新版本是IBIS 2.6。模型包括地表模块、土壤地球生物化学模块、植被物候模块和植被动态模块5个模块,属于动态植被类型模型。IBIS模型综合考虑了地球化学过程、地球物理过程和植被动态过程,能够耦合大气环流模式。模型时间步长为1小时,在空间上适用的尺度包括斑块、景观、区域和全球,属于植被动力学模型(DGVM)。模型的基本信息如表~\ref{tab:model-basic-cmp}第2列,输入输出项如表~\ref{tab:model-io-cmp}第4列,模型的植被功能类型综合考虑了气候区(北方、温带、热带)和植被类型(常绿林、落叶林、针叶林、混交林、草地、灌木等),具体划分类别如表~\ref{tab:PFTs}第3列所示。其他详细参数见Foley等~\cite{foley1996integrated},Kucharik等~\cite{Kucharik2000Testing}

\begin{table}[H]
    \centering
    \caption{IBIS、Biome-BGC和LPJ模型的基本信息对比}
    \label{tab:model-basic-cmp}
    \begin{threeparttable}
        \begin{tabular}{llll}
            \Xhline{1.5pt}
             & IBIS & Biome-BGC & LPJ \\
            \Xhline{1pt}
            类型 & DGVM & 生物地球化学模型 & DGVM \\
            时间分辨率 & 1小时 & 1天 & 1天 \\
            空间分辨率 & $0.5^{\circ}, 1^{\circ}, 2^{\circ}, 4^{\circ}$ & 无明确限制 & $0.5^{\circ}$ \\
            开发语言 & Fortran77 & C & Fortran90 \\
            运行平台 & Windows/Linux & Windows/Linux & Windows/Linux \\
            \Xhline{1.5pt}
        \end{tabular}
    \end{threeparttable}
\end{table}

\subsubsection{Biome-BGC模型}
Biome-BGC~\cite{thornton1998regional}~\cite{kimball1997simulating}~\cite{white2000parameterization}模型由美国蒙拿大大学森林学院陆地动态数值模拟团队(Numerical Terradynamic Simulation Group,NTSG)基于FOREST-BGC模型改进而来,广泛应用于模拟植被、凋落物、土壤中的碳、氮、水等物质的循环过程~\ref{Thornton2001Modeling}。模型使用静态植被类型,属于生物地球化学循环模型,包括的生理生态过程有光合作用、蒸散发、生态系统呼吸、分解、光合产物的分配以及植物的死亡,基本原理是物质和能量守恒定律,即进入生态系统中的物质和能量,等于离开生态系统的与生态系统中累积的之和。模型的基本信息如表~\ref{tab:model-basic-cmp}第3列,输入输出项如表~\ref{tab:model-io-cmp}第5列,模型的植被功能类型被分为7类,具体如表~\ref{tab:PFTs}第4列所示,每种植被功能类型对应一套生理生态参数。该模型在空间上无明确的限制,适用于区域和全球尺度,模拟时间分辨率为1天。

Biome-BGC模型的运行分为两阶段:spin-up阶段是获取模拟的初始状态,第二步是使用第一阶段的结果参数正式运行。初始状态的获取有两种方法,一是直接输入长期定点观测的参数值,二是运行模型自带的spin-up程序。在spin-up过程中,将模拟起点时的碳氮存量设为极低的值(叶片的碳存量为$0.001kgC m^{-2}$,其他库的碳氮存量均为0),反复模拟数千年,直到连续两年土壤碳差值小于$0.005kgC m^{-2}$,或达到设定的最大迭代年数为止,这时认为系统达到稳定状态~\cite{thornton2005ecosystem}~\cite{pietsch2005bgc}。

\begin{table}[H]
    \centering
    \caption{IBIS、Biome-BGC和LPJ模型的输入输出对比}
    \label{tab:model-io-cmp}
    \begin{threeparttable}
        \begin{tabular}{l|l|l|ccc}
            \Xhline{1.5pt}
            \multicolumn{3}{c|}{数据} & IBIS & Biome-BGC & LPJ \\
            \Xhline{1.5pt}
            \multirow{15}{*}{输入} & \multirow{10}{*}{气象数据} & 平均温 & √ & √ & √ \\
            & & 最高温 & √ & √ & \\
            & & 最低温 & √ & √ & \\
            & & 相对湿度 & √ & & \\
            & & 降水 & √ & √ & √ \\
            & & 风速 & √ & & \\
            & & 云量 & √ & √ & \\
            & & 白天时长 & & √ & \\
            & & 短波辐射 & & √ & √ \\
            & & 饱和水气压差 & & √ & \\
            \cline{2-6}
            & \multirow{3}{*}{土壤数据} & 沙粒、粉粒、黏粒含量 & √ & √ & \\
            & & 土壤深度 & & √ &  \\
            & & 土壤类型 & & & √ \\
            \cline{2-6}
            & \multicolumn{2}{l|}{高程} & & √ & \\
            \cline{2-6}
            & \multicolumn{2}{l|}{$CO_2$浓度} & & √ & \\
            \cline{2-6}
            & \multicolumn{2}{l|}{植被功能类型} & √ & √ & \\
            \hline
            \multirow{3}{*}{输出} & \multicolumn{2}{l|}{地球生物化学通量} & √ & √ & √ \\
            & \multicolumn{2}{l|}{潜在植被功能类型} & √ & & √ \\
            & \multicolumn{2}{l|}{LAI} & √ & √ & √ \\
            \Xhline{1.5pt}
        \end{tabular}
    \end{threeparttable}
\end{table}

\subsubsection{LPJ模型}
LPJ(Lund-potsdam-Jena Dynamic Global Vegetation Model,LPJ)~\cite{sitch2003evaluation}模型建立在Biome3模型的基础上,包括的植被生理过程包括光合作用、呼吸作用、植被冠层能量交换、土壤水平衡等。LPJ模型适用于大尺度、中尺度以及全球尺度范围内的碳循环模拟。LPJ模型的时间步长是1月,Venevsky~\ref{venevsky2007sever}对运行时间步长进行了修改,时间步长变成了1天。为了和IBIS模型与Biome-BGC模型的时间分辨率一致,本文应用的是Venevsky的LPJ-Daily(如无特殊说明,本文后面所说的LPJ均指LPJ-Daily)。模型的基本信息如表~\ref{tab:model-basic-cmp}第4列,输入输出项如表~\ref{tab:model-io-cmp}第6列,植被功能类型分为10种,具体如表~\ref{tab:PFTs}第5列所示。LPJ也属于DGVM,不需要输入初始植被功能类型,并且会输出终止年份的潜在植被功能类型。

LPJ模型的运行也分为两个阶段:spin-up阶段假设没有植被(全为裸土地)和生物量,进行1000年积分直到植被覆盖和土壤碳达到平衡态;常规运行阶段使用spin-up阶段运行出的平衡态作为初始值进行模拟。与Biome-BGC不同的是,LPJ模型在内部将两个阶段直接串联在一起,不用分两步运行。

\begin{table}[H]
    \centering
    \caption{IBIS、Biome-BGC和LPJ模型的植被功能类型分类标准}
    \label{tab:PFTs}
    \begin{threeparttable}
        \begin{tabular}{lllll}
            \Xhline{1.5pt}
            大类 & IGBP & IBIS & Biome-BGC & LPJ \\
            \Xhline{1.5pt}
            \multirow{3}{*}{Grass} & GRA & GRA & C3 & C3 \\
            & SAV & SAV & C4 & C4 \\
            & WSA & - & - & - \\
            \hline
            \multirow{2}{*}{Shrub} & OSH & OSH & \multirow{2}{*}{Shrub} & \multirow{2}{*}{-} \\
            & CSH & CSH & & \\
            \hline
            \multirow{3}{*}{DF} & DBF & Tropical & DBF & Raingreen \\
             & DNF & Temperate & DNF & Summergreen \\
             & - & Boreal & - & - \\
             \hline
            \multirow{4}{*}{EF} & EBF & Tropical & EBF & Tropical Broadleaf \\
             & ENF & Temperate Boradleaf & ENF & Temperate Boradleaf \\
             & - & Temperate Needleleaf & - & Temperate Needleleaf \\
             & - & Boreal & - & Boreal Needleleaf \\
             \hline
             & MF & MF & - & \\
             \hline
             & CRO & - & & \\
             \hline
             & WET & - & & \\
            \Xhline{1.5pt}
        \end{tabular}
    \end{threeparttable}
\end{table}

\section{陆地生态系统碳循环相关数据}
~\label{sec:model-data}
% 缺图:a陆面掩膜;b PFT;c DEM
% $CO_2$ 1982-2014 时间序列图
% 缺表:三个模型需要的数据对比表
本文所选的三个模型均以网格点为基本模拟单位,针对全球范围内的区域,本文使用$0.5^{\circ} \times 0.5^{\circ}$的经纬网划分格点,并筛选去除海洋、贫瘠土地、湿地和农田的格点,最终在全球范围内共有40595个格点。

如前文所分析,生态过程模型一般都需要气象数据集、土壤数据集和植被功能类型数据集,而在进行对比时,要确保使用相同的通用数据集。而由于不同模型对这些数据在时空分辨率、植被类型等方面的要求不同,因此,本文采用相同的数据域,针对每个模型的特殊要求,对原始数据进行数据处理。

\subsection{气象数据集}
本文采用MERRA 2(Modern Era Retrospective-analysis for Research and Applications 2)气象数据集,MERRA是NASA为卫星提供的在大气再分析资料,致力于天气和气候时间尺度与水循环相关研究,具有较高的空间分辨率和较完整的时间序列范围,气象要素包括平均温、最高温、最低温、相对湿度、降水量、风速和云量,数据时间范围从1982年到2013年共32年,以日为时间步长。数据原始空间分辨率为$0.625^{\circ} \times 0.5^{\circ}$,对其进行重采样转换为$0.5^{\circ} \times 0.5^{\circ}$。

\subsection{土壤数据集}
主要包括土壤的颗粒组成成分,即沙粒含量、黏粒含量、粉粒含量、土壤深度,由中山大学土壤——大气相互作用研究组开发~\ref{shangguan2014global}。数据的空间分辨率为$0.0085^{\circ} \times 0.0107^{\circ}$,对其进行重采样转换为$0.5^{\circ} \times 0.5^{\circ}$。数据详情参考\href{http://globalchange.bnu.edu.cn/research/soilw}{http://globalchange.bnu.edu.cn/research/soilw}。

\subsection{植被功能类型数据集}
生态过程模型普遍采用植被功能类型作为植被基本处理单元,本文所选的三个模型也是如此,但是这三个模型的分类标准不同,观测数据使用的IGBP的植被分类标准,Biome-BGC的分类最简单,而IBIS和LPJ两个模型则综合考虑了植被类型和气候分区,他们的分类具体如表~\ref{tab:PFTs}。本文使用柯本——盖格尔气候分类标准,该分类系统由K\"{o}ppen和Geiger等发布,综合考虑了温度、降水、气候带和植被等因素,将气候划分为5个主要的气候组:热带、干旱、温带、大陆和极低,共包括31中植被类型,具体如图~\ref{fig:PFT}所示。本文将其重新聚类,图~\ref{fig:PFT-IBIS}和图~\ref{fig:PFT-Biome-BGC}分别表示IBIS和Biome-BGC使用的植被功能类型。


\begin{figure}[!h]
    \centering
    \includegraphics[width=1\textwidth]{PFT}
    \caption{植被功能类型图(具体参考K\"{o}ppen~\cite{peel2007updated})}
    \label{fig:PFT}
\end{figure}

\begin{figure}[!h]
    \centering
    \subcaptionbox{IBIS的PFT\label{fig:PFT-IBIS}}{\includegraphics[width=.45\textwidth]{PFT-IBIS}}
    \hfill
    \subcaptionbox{Biome-BGC的PFT\label{fig:PFT-Biome-BGC}}{\includegraphics[width=.45\textwidth]{PFT-Biome-BGC}} \\
    \caption{重分类后的IBIS和Biome-BGC的植被功能类型}
    \label{fig:carbon-cycling}
\end{figure}

\subsection{通量观测数据集}
% 举例的站点表
全球通量观测网络(FLUXNET)是一个以全球广泛分布的通量塔为基础的全球通量观测网络。通量观测站点遍布世界主要国家,在北美、欧洲、亚洲、非洲都有它的区域网络存在(如亚洲通量网AsiaFlux、美洲通量网AmeriFlux、中国通量网ChinaFlux)。到如今为止,累计有800多个活跃和历史的通量测定站点,分布在全球大部分气候空间和代表性生物群系中。覆盖的植被功能类型包括热带森林、温带针叶和阔叶森林、寒带森林、农田、草原、灌木丛、苔原和湿地。它在全球范围内连续测量大气状态变量,如温度、湿度、风速、降雨量、大气二氧化碳、植被生产力、生物量等。
FLUXNET的通量数据有三个版本,本文使用的是最新的FLUXNET2015数据集,包括温度、降水、光合有效辐射比例(FPAR)、GPP、NEP等通量数据,共计有231个通量观测站点,累计900余个站点$\cdot$年次,排除误差后还有129个站点,其分布如图~\ref{fig:sites-map}所示。

\begin{figure}[!htbp]
    \centering
    \includegraphics[width=1\textwidth]{sites-map}
    \caption{129个FLUXNET观测站点分布}
    \label{fig:sites-map}
\end{figure}

% TODO 重分类
\begin{table}[!htbp]
    \centering
    \caption{站点植被功能类型统计表}
    \label{tab:site-PFT-stat}
    \begin{threeparttable}
        \begin{tabular}{c}
            \Xhline{1.5pt}
            \Xhline{1.5pt}
            \Xhline{1.5pt}
        \end{tabular}
    \end{threeparttable}
\end{table}

% 林地 草地 农田 湿地 灌木 草原 其他

% EBF     Evergreen Broadleaf Forests         常绿阔叶林          -
% ENF     Evergreen Needleleaf Forests        常绿针叶林          -
% DBF     Deciduous Broadleaf Forests         落叶阔叶林          -
% DNF     Deciduous Needleleaf Forests        落叶针叶林          -
% MF      Mixed Forests                       混交林
% OSH     Open Shrublands                     稀疏灌丛            -
% CSH     Closed Shrublands                   郁闭灌丛            -
% WSA     Woody Savannas                      多树草原            -
% SAV     Savannas                            热带稀树草原
% GRA     Grasslands                          草地                -
% WET     Permanent Wetlands                  永久湿地
% CRO     Croplands                           农田

\subsection{MODIS GPP/NPP数据集}
MODIS MOD17 A3提供了时空连续的遥感GPP/NPP产品,MODIS/GPP产品以其他MODIS产品为基础计算而得,时间范围从2000年到2015年,时间分辨率为8天,空间分辨率为$0.0083^{\circ} \times 0.0083^{\circ}$,对其进行重采样转换为$0.5^{\circ} \times 0.5^{\circ}$。本文使用MODIS GPP/NPP 数据集作为全球尺度上的对比验证数据集,与三个模型模拟的结果进行对比分析。

% \subsection{其他数据集}
% 模型还需要网格点的物理常量数据,包括有格点高程数据、短波反射率等。其中高程数据使用的是DEM数据

\section{陆地生态系统碳循环模型对比方案}
\label{sec:cmp-sln-analysis}
% 和国内外研究现状重复?
% 更详细一点:包括会议何年何地开展的,都可以详细介绍
\subsection{基于实验协议的模型对比方案}
由于碳循环模型的复杂性和多样性,目前大规模的模型对比方案都是通过组织开展模型对比计划、制定标准试验协议的方式进行,GAIM、C-LAMP、LBA-DMIP等都是如此。对比研究小组向参与对比的模式组提供驱动数据,并制定非常详细的实验参数,以确保模型运行的驱动条件一样。LBA-DMIP在对比时选择了8个观测站点,并提供了站点上的通量观测数据和土壤、气象、植被和$CO_2$浓度等数据;GAIM则研究全球陆地范围,根据模型分类提供了遥感数据、植被、气象等数据;C-LAMP项目通过与大气碳循环模型耦合来进行对比,设定了对比实验所使用的$CO_2$浓度、氮存量、土地覆盖等参数。模型对比的实验项通常覆盖碳循环的几个关键问题,包括:

\begin{enumerate}[(1)]
\item 陆地生态系统碳循环模型对历史时期植被生长和碳循环的模拟;
\item 陆地生态系统碳循环模型对未来气候变化(温度、降水)和$CO_2$浓度的响应;
\item 陆地生态系统碳循环模型的参数敏感性分析;
\item 陆地生态系统碳循环模型的参数校准。
\end{enumerate}

模式组将模拟结果提交给评审委员会,经过评审委员会统一对结果统计分析后发布出来。

\subsection{可迁移、复用与重现的模型对比方案}
% CORDEX OCW ILAMP
% 基于虚拟机镜像/包管理器/docker
碳循环的模拟在时间和空间上是多尺度的,在不同地区、时间范围和时空尺度上模拟时,基于实验协议的模型对比方案往往难以重用。
为了解决这一问题,国际土地模型基准(ILAMB,THE INTERNATIONAL LAND MODEL BENCHMARKING PROJECT)开发了一套模型评价分析工具,并将其发布为开源的python软件包。ILAMB评估的统计学指标包括时间和空间上的均值、误差、均方根误差、年际变率等,同时还提供泰勒图、折线图、地图、相关性分析等功能,ILAMB综合了这些评价指标定义了一套模型评价体系,对参与对比的模型评分。

ILAMB提供的第三方软件包极大地简化了对比的过程,但是由于软件环境的复杂性,在实际使用时很难部署。区域气候模型评估系统(RCMES,Regional Climate Model Evaluation System)开发了一套模型对比软件资源,并将其发布为conda软件包,同时还提供了虚拟机镜像环境,在一定程度上解决了模型对比时难以迁移、复用和重现的问题。

\subsection{陆地生态系统碳循环模型对比方案总结}
% 这一部分很重要,要明确的说明以上方案的不足之处
% 引出我们的对比方案的优点和设计
% 这是立题之本
% 明确说明
%   对比过程中共享与重用的需求
%   对比流程

综上所述,基于实验协议的对比方案针对碳循环模型研究的关键问题,制定了非常细致的对比实验,但是,这种对比方案以模型运行结果数据为中心,不考虑生产数据即模型运算的过程。而事实上模型对比是涵盖数据生产、传输、处理、分析的一整套流程,将模型运算剥离开来在某种程度上使对比流程不够开放、透明;面对对比方案的重用性,基于软件开发包和虚拟机镜像的对比工具虽然降低了软件迁移、复用和重现的难度,但存在一定的复杂性。

\section{本章小结}
% TODO
% 本章首先分析了陆地生态系统碳循环的基本过程,接着总结了陆地生态系统碳循环模型的分类体系和基本结构,并详细介绍了本文所对比的三个生态过程模型:IBIS、Biome-BGC和LPJ;最后针对这些模型所需要的数据做出了详细的介绍。