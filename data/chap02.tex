\chapter{基于Web Service的开放式模型对比框架研究}
% - 章节意义:...是...的重点/前提,...有重大意义。
% - 本章内容:本章首先分析/阐述...,进一步...;接着探讨...;最后讨论...。针对框架构建的关键问题,分析现有框架设计的方法及其存在问题,提出了本文的重点内容。

模型对比框架的设计是填充模型资源和数据资源,开展详细对比工作的前提。分析现有的对比框架可以为开放式的模型对比框架提供参考和借鉴,找出模型对比框架设计的重点。

本章首先分析了现阶段国内外主流的模型对比框架设计方法,进一步总结了现阶段对比方案的优缺点;接着探讨了在开放式网络环境下如何设计公开化可参与的对比框架;最后讨论了在开放式的模型对比框架下可以接入的外部资源库,丰富了对比系统的内容。针对构建开放式地理模型对比系统,提出了本文的重点内容。

\section{模型对比框架分析与总结}
\subsection{基于实验协议的框架}
% CMIP

\subsection{可迁移与复用的框架}
% CORDEX
% 基于虚拟机镜像/包管理器/docker

\subsection{现有框架的分析和总结}
传统的模型对比框架是以数据为基础,不考虑生产数据即模型模拟的过程。

\section{开放式模型对比框架设计}



\section{开放式模型对比外部资源库}

\subsection{模型资源库}
% 介绍模型
\begin{enumerate}[(1)]
\item \textbf{IBIS}

\item \textbf{Biome-BGC}

\item \textbf{LPJ}

\end{enumerate}


\subsection{数据资源库}
% 介绍数据
\begin{enumerate}[(1)]
\item \textbf{气象数据集}

\item \textbf{土壤数据集}

\item \textbf{Fluxdata观测数据集}

\item \textbf{MODIS GPP产品数据集}

\item \textbf{模型输出数据集}

\end{enumerate}

\subsection{对比方法库}
% 介绍对比方法
\begin{enumerate}[(1)]
\item \textbf{泰勒图}

\item \textbf{时间序列折线图}

\item \textbf{箱图}

\item \textbf{热力图}

\item \textbf{偏差等值线图}

\item \textbf{加权超级集合}

\end{enumerate}


\subsection{数据处理方法库}

\subsection{可视化方法库}

\section{本章小结}