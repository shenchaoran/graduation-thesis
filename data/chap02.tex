\chapter{陆地生态系统碳循环模型对比场景分析}
% - 章节意义:...是...的重点/前提,...有重大意义。
% - 本章内容:本章首先分析/阐述...,进一步...;接着探讨...;最后讨论...。针对框架构建的关键问题,分析现有框架设计的方法及其存在问题,提出了本文的重点内容。

% 不知道哪里改写具体的碳循环模型和数据,哪里该写广义上的地理模型和地理数据

模型对比场景是开展模型对比工作的承载容器。在地理模型对比场景中,模型资源和数据资源作为最重要的两个要素,为开展详细的对比工作准备。
% 模型对比框架的设计是填充模型资源和数据资源,开展详细对比工作的前提。分析现有的对比框架可以为开放式的模型对比框架提供参考和借鉴,找出模型对比框架设计的重点。

本章首先分析了地理模型资源和数据资源的多源异构复杂性和这种复杂性对开展模型对比工作的阻碍;然后分析和归纳了现阶段国内外主流的模型对比方案设计方法;最后探讨了在开放式网络环境下如何设计公开化可参与的对比框架,并讨论了模型对比过程中几个至关重要的地理资源库。

\section{陆地生态系统碳循环模型及其特征分析}
\subsection{陆地生态系统碳循环模型及其分类}
\subsection{陆地生态系统碳循环模型的特点}
\subsection{IBIS, Biome-BGC, LPJ模型简介}
\section{陆地生态系统碳循环数据资源及其特征分析}
% \subsection{陆地生态系统碳循环模型相关数据}
% \subsection{数据特征分析}
\subsection{气象数据集}
\subsection{土壤数据集}
\subsection{其他输入数据集}
\subsection{通量观测数据集}
\section{陆地生态系统碳循环模型对比方案分析和归纳}
\subsection{基于实验协议的模型对比方案}
% CMIP

\subsection{可迁移、复用与重现的模型对比方案}
% CORDEX
% 基于虚拟机镜像/包管理器/docker

\subsection{陆地生态系统碳循环模型对比方案总结}
传统的模型对比框架是以数据为基础,不考虑生产数据即模型模拟的过程。

\section{本章小结}