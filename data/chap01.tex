% 开放式地理模型对比系统构建方法研究——以陆地生态系统碳循环模型为例
\chapter{绪论}

\section{研究背景及意义}

\subsection{研究背景}
% - 介绍地理模型。介绍模型对比的意义。介绍传统的地理模型的对比,面临的问题。
% - 碳循环模型,作用、意义、发展,和对比现状
% - 随着时代的发展涌现了一些新技术,为解决GIS领域的问题提供了新的视角和思路。介绍 web service、模型共享、数据共享、对比方案的共享等。
% - 总结开放式对比的难点、意义


% 传统的地理模型对比通常是将参与对比的模型统一部署在本地环境下,通过配置对应的输入数据驱动模型的运行,运算结束后对输出结果进行数据处理,最后进行对比分析。

%自工业革命以来,人类活动正在大规模地改变陆地生物圈结构,其中最典型的是温室气体浓度尤其是CO2浓度的增加导致全球气候变暖~\cite{hallett2002climate}。陆地生态系统碳循环模型是估计和预测不同尺度碳收支格局和变化的重要手段~\cite{cao2003interannual}。

地理模型是对地理过程和机理的抽象表达,是通过模拟、预测或重现这些地理现象来解决地理问题的重要手段。通过地理模型的模拟可以有效解决复杂的地理问题,促进地理学研究的发展。地理模型的对比是检验和验证模型模拟效果的必要方式,可以有效促进模型的发展改进。

以陆地生态系统碳循环领域为例,为了模拟生态系统中大气、海洋、陆表和化石燃料四个碳库之间的碳循环过程,预测不同尺度的碳收支格局和变化情况(Cao M K et al., 2005;Cao,et al.,2003),大气学家研发了众多的碳循环模型,大致可以分为统计模型、遥感参数模型、生态过程模型和遥感、过程耦合模型四类。代表性的有Miami模型~\cite{Lieth1975Primary}、CASA~\cite{Potter1999Interannual}~\cite{Potter2003Continental}、GLO-PEM~\cite{Prince1995Global}~\cite{Goetz2000Interannual}、BIOME-BGC~\cite{running1988general}~\cite{Running1991FOREST}~\cite{thornton2000user}、IBIS(Foley et al., 1996)、LPJ DGVM~\cite{Gerten2004Terrestrial}~\cite{Sitch2010Evaluation}等。各个模型由于其模拟机理不同,有其各自适用的尺度和范围。

% 全球气候研究计划(WCRP,World Climate Research Programme)开展了6次耦合模型对比计划(CMIP,Coupled Model Intercomparison Project),旨在更好地理解过去、现在和未来的气候变化,它强调分享、比较和分析全球气候模式成果的重要性,并提供高质量的气候信息,作为气候评估和谈判的基础(Gerald A. Meehl, 2000)。

为了全面客观地检测气候变化,评估人为因素对气候变化的影响,以及更好的模拟气候变化,国际上成立了联合国政府间气候变化专门委员会(IPCC,Intergovernmental Panel on Climate Change)。IPCC每5年开展并出版一次评估报告,提供有关气候变化的现象、成因、影响以及对策等信息~\cite{Oliver2013Intergovernmental}。
除此之外,世界气候研究计划(WCAR,World Climate Research Programme)成立了耦合模型对比计划(CMIP,Coupled Model Intercomparison Project),针对每届IPCC评估报告中出现的科学问题制定标准实验方案,对比各个模型的模拟能力~\cite{meehl2000coupled};
其中,第六阶段耦合模型对比计划(CMIP6)背书的模型对比项目有23余个,包括碳循环模型对比项目、云反馈对比项目、十年际气候预测项目、海洋模型对比项目、古气候模型对比项目、辐射强迫模型对比项目、协调区域降尺度实验等。~\cite{WCRP-CMIP6-Endorsed-CMIPs}
可见,模型对比评价是当前气候模型研究领域的一个\textbf{热点}。

% 区域气候模型评估系统(RCMES,Regional Climate Model Evaluation System)和协调区域气候降尺度实验(CORDEX,Coordinated Reginal Climate Downscaling Experiment)也在区域尺度上开展模型对比评估,通过区域尺度上的评估决策弥合全球尺度上的气候变化~\cite{loikith2013scientific}。

据统计,在第五次耦合模型比较计划(CMIP5)中,超过20个建模组使用50多个碳循环模型参与模拟~\cite{Taylor2012An},越来越多的模型参与极大地促进了对气候变化的监测和评估。但是,模型对比仍然面临着诸多困难。
第一,由于地球系统模式的复杂性,碳循环模型通常涉及到生物(光合作用、呼吸作用)、物理(植被冠层能量交换)、化学(碳、水循环)、遥感(数据同化)等模块,每个模型由于其机理不同,涉及到的模块组成往往不同,而且各个模块的实现方法也不尽相同;
第二,由于地理现象的时空分异特性,地理模型在模拟地理现象时往往受到研究区域的局限,对于模型全面的评估应该是从多尺度、多视角进行;
第三,由于模型参数的复杂性,在应用模型模拟时,在调参方面需要花费巨大的精力,调参过程也需要耗费大量的计算资源;
第四,碳循环模型由不同的科研人员编写,在实现层次上,面临着操作系统、编程语言、软硬件环境依赖、运行方式、输入输出数据格式等多方面的异构性,难以部署、编译和应用,阻碍了模型的推广。
这些原因共同限制了模型对比工作的发展,模型对比也是当前气候模型研究领域的一个\textbf{难点}。

随着计算机技术的发展,Web Service和面向服务的软件架构(SOA,Service-Oriented Architecture)技术不断成熟,为分布式网络环境下的地理模型的共享与重用提供了技术上的有效支持,地理模型和数据逐渐朝着网络服务化的方向进行封装与共享~\cite{胡迪2015地理模型的服务化封装方法研究}。服务化的地理模型将复杂的地理模型封装为一个黑箱,屏蔽了模型在使用平台、编程语言、软硬件环境依赖以及数据格式等方面的复杂性和异构性,使用者不必关心其内部实现的细节,简单了解其调用接口就能够开展地理模拟;而地理数据服务促进了数据的共享重用,降低了模型应用时数据搜集的困难~\cite{Yue2015A}。以地理模型服务和数据服务为基础,将网络服务技术手段应用到地理模型对比上,不仅可以将对比过程公开化透明化,而且可以将对比方案共享出来,达到可重用的目的,为开放式模型对比提供了一种新的视角和技术手段。\textbf{本文基于Web Service技术规范,设计了一套开放式模型对比框架,可以支持开放式网络环境下异构地理模型的参与式和可共享式对比。并以IBIS、LPJ、Biome-BGC三个开源碳循环模型为例,模拟了三个模型在全球150个站点上的碳收支排放,并与Fluxdata和MODIS观测数据进行对比,评估了三个模型的模拟能力,验证了开放式模型对比框架的可用性,为地理模型对比提供了一种新的思路。}


\subsection{研究意义}
% - 通用的意义:开放、公平、可共享、可重用。
% - 对碳循环模型的意义:促进碳循环模型的发展改进,提高碳循环模型的精度,提高人们对碳排放的认知,为解决气候问题提供准确有效的模拟手段。
采用传统的单机式地模型部署和对比需要将所有模型部署在同一台计算机上,面临着部署过程复杂、对比结果难以再现、对比方法不公开等一系列问题。本文通过基于Web Service的对比框架设计,对地理模型来说,屏蔽了其在操作系统、编程语言、软硬件环境依赖以及数据格式依赖等方面的复杂性和异构性,降低了模型的使用门槛;对地理数据来说,降低了数据收集、存储、管理、处理的繁琐流程;对于对比方案来说,一方面可以将传统的线下对比过程公开化、透明化,使对比过程具有可追溯性、可验证性和易重复性,另一方面对比方案也可以共享出来,达到可共享、可复用的目的。能够提供一个更加公平公正公开的“擂台”。

对于陆地生态系统碳循环领域,本平台提供了一个开放式的模型对比“擂台”,通过碳循环模型之间的对比,一方面验证和诊断了模型的模拟效果,促进了模型的应用和改进,另一方面促进了耦合模型的协同研究和实验,从而在人类活动对自然环境的影响方面获取更可信地评判,指导人类面对气候问题及时做出正确的决策。

\section{国内外研究现状综述}
% 碳循环模型的对比

开放式地理模型对比系统的构建涉及多个方面,包括模型和数据资源的深入了解、模型对比评价方法的总结归纳、以及开放式服务化系统架构的搭建。认识地理模型的运行行为特征和数据特征是驱动其模拟计算的前提,剖析前人对比领域模型的统计学方法是丰富对比形式的手段,熟悉开放式服务化系统的框架是构建系统的基础。因此,本文从陆地生态系统碳循环模型的发展历程,前人对碳循环模型的对比评价方法以及开放式服务化系统的框架三个方面进行回顾,分析陆地生态系统碳循环领域地理模型对比的研究现状、存在问题和发展趋势。

\subsection{陆地生态系统碳循环模型}
% 模型机理介绍

陆地、大气、海洋以及化石燃料碳库是地球系统碳循环的四个主要组成部分,其中陆地生态系统又是其中最为活跃的碳库,也是人类活动聚集的场所。陆地生态系统碳循环模型通过模拟植物光合器官碳库、植物支持器官碳库、凋落物碳库和土壤有机碳库之间的碳源汇交换,来计算植被的初级生产力~\cite{毛留喜2006陆地生态系统碳循环模型研究概述}。

陆地生态系统碳循环模型从机理上可分为统计模型、生态过程模型和遥感、过程耦合模型。统计模型可分为气候统计模型和遥感统计模型。其中气候统计模型主要通过在气候因子与植被净初级生产力(Net Primary Productivity,NPP)的实测数据之间建立回归方程;遥感统计模型通过遥感光谱指数(如NDVI)与NPP、生物量等数据间的相关关系进行统计回归。统计模型简单直观,具有较强的区域适用性,但其完全依赖于地面观测数据,对于不同的区域,模型不具备普适性和推广性。同时统计模型没有考虑陆地生态系统碳循环过程的内部机理,无法揭示生态系统与环境间的相互影响关系,不能用于对未来的预测研究~\cite{袁文平2014陆地生态系统植被生产力遥感模型研究进展}~\cite{谢馨瑶2018大尺度森林碳循环过程模拟模型综述}。生态过程模型按照是否考虑实际环境对植被类型、组成和结构的影响分为基于动态植被类型的模型和基于静态植被类型的模型~\cite{王绍刚2008森林碳循环模型方法研究进展}~\cite{王萍2009森林碳循环模型概述}~\cite{毛留喜2006陆地生态系统碳循环模型研究概述},按照涉及到的机理类型分为地球化学过程模型、陆面物理过程模型和生物过程模型~\cite{谢馨瑶2018大尺度森林碳循环过程模拟模型综述}。过程模型由于其综合考虑了碳循环过程的动力学特征,结合了气候、土壤和植被生理生态参数,以及陆地生态系统与大气、海洋之间的相互作用,模拟结果相对来说更加准确,逐渐占据了主导地位。遥感、过程耦合模型通过将遥感观测数据(如叶面积指数LAI)同化到模型之中,来提高模型模拟的精度~\cite{2013基于数据同化的哈佛森林地区水、碳通量模拟},他融合了遥感统计模型和生态过程模型的优点,可以反映区域和全球尺度的NPP空间分布和变化~\cite{朱文泉2005陆地植被净初级生产力计算模型研究进展}。

综上所述,生态过程模型在全球范围内不仅模拟效果好,而且与遥感、过程耦合模型相比更加简单、容易实现。因此,本文选取了三个适合全球尺度的生态过程模型参与对比:IBIS、BIOME-BGC、LPJ DGVM,并总结了各自的特征,如表1所示。从中可知,这些模型的时空适用范围、输入数据、输出要素都很相似,均具备全球范围内碳循环的模拟能力,彼此之间可以进行对比。


\subsection{陆地生态系统碳循环模型的验证与对比}
% - 

对于陆地生态系统碳循环模型模拟结果的验证和对比,前人大多从多个视角进行综合对比,包括对模拟结果的可视化观察对比,使用统计学方法定量地对比等。没有单独的评价手段被认为是优越的,相反的,多种对比技术的结合使用能够为模型模拟能力提供全面评价~\cite{Taylor2012An}。

在对比对象方面,目前主要分为三类:基于通量站点数据的对比、基于卫星遥感数据的对比和基于多模型模拟结果的交叉对比。杨延征~\cite{杨延征2016基于}、王磊(2010)、刘曦~\cite{刘曦2011IBIS}、张海燕(2006)等人分别在中国全国尺度、中国东部、东北东部森林和东北帽儿山应用IBIS模型估算碳排放情况,将模型模拟的结果与通量站点观测数据对比,验证其在中国的模拟能力。胡瀞予(2011)、NIU Ben~\cite{Niu2017Satellite}等人分别以台湾陆域生态区和青藏高原为研究区域,将生物测定法、Rahman’s方程式以及VPM、PCM、AVM的模拟结果与MOD17数据集相对比,评估其模拟能力。P.Friedlingstein(2006)在CMIP4中通过11个碳循环模型之间模拟结果的相互比较来评价其模拟能力。

在对比方法方面,通过绘制模拟值等值线图、观测值等值线图、差异等值线图可以非常直观地展示出各个模型模拟结果的差异;通过泰勒图展示模拟值与观测值之间的标准差之比、均方根误差以及相关系数;通过纵向图和时间序列折线图可以清晰地展现模型在各个子区域中的适用情况~\cite{Kim2013Evaluation}~\cite{Kim2014Evaluation}~\cite{Kim2017Winter}~\cite{刘敏200913}(Kim J,2012;Kim J,2014;Kim J,2018;郭彦,2013;刘敏,2009)。Stoner A M K~\cite{Stoner2013An}使用统计降尺度的方法建立当前观察值、当前模拟值、未来模拟值与未来观察值之间的关系,预测未来的气候因子发展状态。

在对比框架方面,国际上比较知名的是WCAR开展的CMIP。通过制定一系列证明模型模拟能力的标准实验,参与比较的模型都遵循实验协议进行测试。使用一致的预测因子(如数据源、空间分辨率、时间分辨率等),针对特定的预测指标(如温度、湿度、降水量、NPP等),在一致的时间范围内模拟,最后将模拟结果提交给专家评审,通过审核的对比结果被公开发布在网站上~\cite{Taylor2012An}。耦合模型对比计划、协调区域降尺度实验以及区域气候模型评估系统都遵循着这种对比框架开展对比~\cite{eyring2016overview}~\cite{Gutowski2009The}~\cite{赵宗慈2016CMIP6}。


\subsection{开放式服务化系统框架}
随着计算机技术的发展,越来越多的数据被发布为服务,其中具有代表性的是开放地理信息联盟(OGC,Open Geospatial Consortium)的WMS、WFS、WCS。在地理模型和数据处理方法方面,OGC WPS(Web Processing Service)也为地理模型的共享提供了解决办法。 %~\cite{Castronova2013Models}

袁爽~\cite{袁爽2010空间数据}基于OGC WFS、WMS标准,实现了空间数据的服务发布;吴楠~\cite{吴楠2012基于}、毛曦~\cite{毛曦2012基于}、宋东泽~\cite{宋东泽2015一个生态传感网的}基于OGC WPS标准,分别设计并实现了VPM模型和日均蒸散量模型的服务发布。Yue S~\cite{Yue2013Key}~\cite{Yue2015A}基于Web Service提出了一套模型服务的元数据描述、封装、打包和发布的技术体系。

\subsection{研究现状分析与总结}
纵观陆地生态系统碳循环模型的研究现状,陆地生态系统碳循环模型的理论体系越来越完善,应用越来越广泛。随着IPCC等国际组织的牵头推动,模型的对比和评价也逐渐开始开展,极大地促进了模型的改进和发展。分析当前的研究现状总结为以下几点:

\begin{enumerate}[(1)]
\item \textbf{陆地生态系统碳循环模型:}

随着技术手段的不断成熟,越来越多的模型被开发出来。但由于地球系统的复杂性,模型在类别、机理、模块结构、参数、数据等方面形态各异,这种多样性阻碍了模型在不同区域以及全球范围内的应用推广,开展模型对比是解决该问题的有效技术手段;

\item \textbf{碳循环模型的对比:}

碳循环模型的对比是当前的研究热点和难点,在对比对象方面模型可以与通量观测站点、遥感影响影像数据以及其他模型的模拟结果相对比;在对比方法方面大多数采用可视化方法和统计学手段;在对比框架方面世界气候研究计划制定了一套模型对比的实验方案,参与者通过实验内容的实施、提交、评审,最终可以将对比结果公开出来。目前的对比相对完善,但是在互联网技术的高速发展下,有待进行一些新的技术尝试;

\item \textbf{开放式服务号系统框架:}

服务号包括模型资源的服务化和数据资源的服务化。目前基于OGC的WFS、WMS和WPS标准形成了一套成熟的技术体系。模型服务和数据服务地发布,可以降低模型的使用难度,促进模型的应用和完善,但是这种技术手段在气候模型以及模型的对比方面缺乏广泛地应用。
\end{enumerate}

\section{研究目标与研究内容}

\subsection{研究目标}
针对目前地理模型和数据资源分散、模型安装部署困难、模型模拟效果难以评估、评估成果难以再现的问题,本文以陆地生态系统碳循环模型为例,基于Web Services技术,旨在设计并实现一个提供模型对比的“擂台”,并用IBIS、BIOME-BGC和LPJ三个模型验证平台的实用性。具体研究目标包括:
(1)	设计出一套新的基于网络服务的开放式地理模型对比系统框架;
(2)	以陆地生态系统碳循环模型为例,实现地理模型资源和数据资源的开放式接入; 
(3)	以陆地生态系统碳循环模型为例,实现对比方案的构建和实施。


\subsection{研究内容}
为实现上述研究目标,设计研究内容如下:
\begin{enumerate}[(1)]
\item \textbf{开放式地理模型对比系统框架的构建}

开放式对比系统框架是开展详细对比工作的基础支撑,它描述了模型资源、数据资源、对比方法资源的标准接入方式,所有这些资源在框架的规范下以科学工作流的方式,有条不紊地运行、耦合、集成。

考虑到网络资源的传输成本、模型运算和模型对比的业务逻辑,本文将整体服务器框架分为三部分:1)资源服务器:负责数据存储、数据库管理、数据服务发布和数据缓存;2)模型计算服务器:负责模型服务发布、模型调用、数据处理和分布式计算;3)模型对比服务器:负责资源汇总和展示、对比方案的构建、对比任务的分发和比较结果的可视化。三个部分之间通过Http和Ftp协议通信。这种情况下,地理模型、地理数据、数据处理方法、数据可视化方法和数据对比方法都以组件的形式分别插入到各个模块中,实现了系统架构的解耦。

\item \textbf{开放式地理模型资源的接入}

地理模型资源是参与模型对比的主体,通过模型资源的汇聚形成模型资源库,而模型资源库是模型对比的重要组成部分。为了实现模型资源的开放式接入,首先需要定义一套模型的元数据描述接口,其中包括模型的属性描述、模型的调用接口描述和模型的软硬件依赖环境描述,结合Web Service技术,通过应用服务器向外暴露Restful API可以将地理模型发布为服务,模型服务向数据库中注册可以实现模型服务的管理。在本文中发布的模型条目包括IBIS、BIOME-BGC、LPJ DGVM。

\item \textbf{开放式地理数据资源的接入}

地理数据资源的接入是驱动模型运行的基础。对于地理数据资源的接入方法和模型资源类似。对于本文涉及到的碳循环模型的运行,需要发布的数据服务包括陆面掩膜数据、气象数据、土壤数据、植被数据和其他配置数据。其中气象数据的时间分辨率是1天,气象参数包括最高温、最低温、日均温、降水量、相对湿度、平均风速、大气压和云覆盖率等;土壤数据包括沙粒含量和黏土含量等;植被数据包括植被类型、凋落物储量和植被功能型数据等。

\item \textbf{地理模型对比方案的构建和实施}

由于单一的对比方法不能够完备地展示出各个模型之间的差异,本文的对比从构建对比方案入手。对比方案是从多个视角综合对比几个模型的模拟能力,是多个对比方法的组合。因此,在构建对比方案之前,首先要构建一个对比方法库,包括现阶段流行的偏差等值线图、泰勒图、时间序列折线图、纵向图、区域降尺度等统计学方法都作为对比方法库的成员,每种对比方法针对特定的结果数据集作出评价。最后,对比方案通过配置具体的输入数据开始模拟,模拟结束会生成一个评价报告文档。在这个过程中,对比方案也作为一种开放的资源共享和重用起来。
\end{enumerate}


\section{研究方法与技术路线}

\subsection{研究方法}
\begin{enumerate}[(1)]
\item \textbf{文献分析法}

搜集并整理国内外相关领域研究者关于陆地生态系统碳循环模型、模型对比方法、模型和数据共享与服务化方法的研究成果,总结前人研究方法的特点和使用范围,针对这些成果的优劣势进行分析和借鉴,为本文的研究方法提供思路。

\item \textbf{归纳演绎法}

分析地理模型在运行行为上的特征,对比并归纳其调用方式,在此基础上设计模型的元数据描述接口,为模型的服务化方法做准备。

\item \textbf{数理统计法}

本文对模型模拟结果的对比评价分析是通过在模型模拟结果与实测值之间进行统计分析,根据统计特征值来评价模拟结果的优劣。

\item \textbf{案例验证法}

本文以IBIS、BIOME-BGC、LPJ模型为例,对各个模型进行封装,以满足模型在站点和区域上的模拟计算,并对模拟结果进行数理统计,分析其优缺点。
\end{enumerate}


\subsection{技术路线}
根据本文的研究目标和研究内容,本文设置的技术路线如图1所示,包括5个阶段:1)问题分析与资料整理阶段。收集并整理文献,分析现有陆地生态系统碳循环模型的特征、碳循 环模型的对比方法和开放式服务化系统架构,确定本文的研究思路;2)模型对比系统框架的构建:搭建系统的整体网络架构,设计各个服务器模块的功能职责和服务器之间的通信方式;3)模型和数据资源的开放式接入:根据模型资源和数据资源的特点,分别设计其开放式资源接入方法;4)模型对比方案的构建和实施:首先丰富模型对比方法库,然后根据模型对比方法的组合和综合评价标准构建模型对比方案,配置数据后调用模型生成评价对比报告;5)实验验证:根据选取的三个模型分别在全球陆地范围内进行模拟,对比其模拟水平。

\section{论文组织结构}
