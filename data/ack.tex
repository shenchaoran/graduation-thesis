% 如果使用声明扫描页,将可选参数指定为扫描后的 PDF 文件名,例如:
% \begin{acknowledgement}[scan-statement.pdf]
\begin{acknowledgement}
时光飞逝,又到了毕业说再见的时候,回首三年紧张且充实的学习生活,往事片段历历在目,给了我人生美好的回忆。值此论文完成之际,谨向给予我教导、支持和关心的老师、同学、家人表示由衷的感谢。

首先,衷心感谢我的导师陈旻教授。陈老师知识渊博、治学严谨、为人谦逊、精益求精,不管是生活还是学术研究上都给了我明确的指导。在论文撰写和系统构建期间,陈老师不仅给了我明确的方向,而且还不辞劳苦,为我反复修改。和陈老师学习的不仅是专业知识和学术工作,更是如何做人做事,这将使我终生受益。

感谢温永宁老师、宋志尧老师、袁文平老师和乐松山大师兄,四位老师的教诲使我时刻铭记在心,四位老师从论文的选题、开题报告的撰写以及论文的写作都亲力亲为,给了我极大的帮助。温老师渊博的知识和超强的编程能力使我非常敬佩,温老师为人随和,平易近人,具有专业的知识和开发经验,在系统架构的构建方面给了我强有力的指导;宋志尧老师学识渊博、为人亲和,不仅是学术上的导师,也是人生路上的导师;袁文平老师在气候变化和碳水循环领域具有很专业的知识,在模型和系统构建方面的给了我非常多的建议和意见;乐松山大师兄不管是在学术研究还是编程技术上都是我们的榜样,在日常工作生活上给了我们很多指导。

感谢地理科学学院和虚拟地理环境教育部重点实验室的所有老师们!感谢江南老师、沈陈华老师、胡迪老师、李硕老师、曹敏老师、韦玉春老师。感谢你们在专业知识方面给我的指导和学习生活方面的诸多帮助。

感谢实验室的所有师兄、师姐、同门、同学、师弟、师妹在学习和生活上对我的帮助:郑培蓓、彭国强、卢付强、陈鹏、张丰源、王进、李玉婷、马载阳、谭羽丰、朱串串、李文沛、武子鸣、夏雪飞、许如琪、刘丹阳、侯涛、张博文、陈坤、王明、孙菱志、汪灵珊、校大卫、芦宇辰、宋杰、王欣畅、潘精明、许凯、张北辰、兰振旭、胡薇等。感谢中山大学李施华师妹、郑艺师姐在模型和数据方面对我的指导。还要感谢我的室友王凯亮、刘章聪、陈博在生活上对我的帮助。

感谢父母的养育之恩和默默支持,使我在物质上无后顾之忧,在精神上有所依靠。正是家人的支持使我能够不断坚持下去,每当面临困难和抉择的时候,是你们为我排忧解难,使我能够坚持下去。

最后向文中所引用的学术研究、论文著作和科研成果的中外学者致谢。感谢参与本文答辩的所有老师。学海无涯,毕业是新的启程!

\begin{flushright}
沈超然 \hspace{.4cm}

2019年5月

于南师仙林 \hspace{.05cm}
\end{flushright}

\end{acknowledgement}
