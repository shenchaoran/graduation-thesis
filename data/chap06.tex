\chapter{结论与展望}
\section{结论}
% 系统方面的结论:系统的可用性;可扩展性
% 陆地碳循环方面的结论
在碳循环模型之间开展模型对比计划是当前研究的热点和难点,采用传统的基于实验协议的对比方案和提供开源对比工具软件资源的方案都存在着一定的缺点。本文以陆地生态系统碳循环模型在网络环境下的开放式对比为研究目标,结合最新的互联网技术,以地理模型服务、地理数据服务、模型对比服务的发布和集成为切入点,设计了开放式、可共享、可重用的碳循环模型对比系统。本文的研究结论如下:
\renewcommand{\labelenumii}{\theenumii}
\renewcommand{\theenumii}{\theenumi.\arabic{enumii}.}
\begin{enumerate}[(1)]
    \item \textbf{开放式对比框架:}通过对模型对比场景和业务逻辑的分析和总结,得出模型对比的难点在于模型资源和数据资源的复杂异构性,导致在本地环境下独立的研究者很难完成多个模型的编译、部署、参数调整、大数据量运行和结果分析等一系列过程。同时也会使对比过程封闭不透明,导致模型的孤岛效应,阻碍模型的应用和推广。而开放式的对比框架允许模式组开发者将模型按照统一的接口接入进来。本文从对比业务、资源组件、网络架构和执行引擎四个角度设计的对比框架实现了以下优点:
        \begin{enumerate}[a)]
            \item 模型计算和对比过程可共享与重用;
            \item 模型和数据资源可以动态接入;
            \item 大规模计算场景下稳定可用;
            \item 以工作流驱动的自动化执行保证了其易用性。
        \end{enumerate}
    \item \textbf{开放式对比资源接入方法:}对比资源包括数据、模型和对比方法脚本三种。其中数据资源的接入包括海量异构数据的重构服务、WMS/WFS/WCS和上传下载服务,数据的开放式接入为模型运行提供基础支撑;模型资源的接入屏蔽了碳循环模型在编程语言、操作系统、软硬件环境、输入输出接口等方面的异构性,降低了模型的使用难度;对比方法的接入针对模型运行结果进行统计学和可视化对比分析。这些资源的开放性接口保证了系统的可扩展能力和公开透明性。
    \item \textbf{IBIS、Biome-BGC、LPJ模型的验证和评价:}三个模型与FLUXNET观测站点和MODIS陆地产品的对比分析得知,三个模型模拟的结果在分布趋势上整体一致,但Biome-BGC的GPP模拟结果在均值、相关性、误差等方面都最好,相比之下LPJ的模拟结果整体偏低,而IBIS的模拟结果整体偏高。
\end{enumerate}

\section{主要的创新点}
本文将高速发展的计算机技术与地理信息系统相结合,面对传统的本地线下碳循环模型对比的需求和痛点,设计的碳循环模型对比框架具有公开性、开放性、共享性、可重用性特征,为碳循环模型对比提供了新的思路。

\section{展望}
本文针对传统的碳循环模型对比框架所面临的难点,探索了开放式网络服务在模型对比方面的应用,设计了模型对比框架,并以IBIS、Biome-BGC和LPJ三个模型证明了开放式对比框架的可用性。然而,本系统在实验应用时受限于模型和数据资源的收集,在资源收集和处理方面不够完善。因此,接下来的工作将从以下几个方面展开:

\begin{enumerate}[(1)]
    \item 由于碳循环模型的复杂性和模拟地区的多样性,碳循环模型在具体应用时可能会面临着更详细的参数调整的需求,比如植被功能类型和土壤类型对应的参数字典等。在本文的模型资源服务化描述和封装方法中需要更详细地对参数列表进行设计,以满足这种需求,保证模型运行的准确性;
    \item 针对碳循环模拟的未来预测情景、敏感性分析情景等方面开展对比实验,从这些情景中更详细地对比模型的模拟能力;
    \item 使用本文设计的模型封装接口封装更多主流的陆地生态系统碳循环模型,加入到对比实验中;
    \item 对系统的进一步调试,使其在功能上更加稳定可用。
\end{enumerate}